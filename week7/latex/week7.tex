\documentclass{article}

\usepackage{fancyhdr}
\usepackage{extramarks}
\usepackage{amsmath}
\usepackage{amsthm}
\usepackage{amsfonts}
\usepackage{tikz}
\usepackage[plain]{algorithm}
\usepackage{algpseudocode}
\usepackage{graphicx}
\usepackage{gensymb}
\usepackage[framed,numbered,autolinebreaks,useliterate]{mcode}
\usepackage{listings}
\usepackage{hyperref}

\graphicspath{{./images/}}

\usetikzlibrary{automata,positioning}

%
% Basic Document Settings
%

\topmargin=-0.45in
\evensidemargin=0in
\oddsidemargin=0in
\textwidth=6.5in
\textheight=9.0in
\headsep=0.25in

\linespread{1.1}

\pagestyle{fancy}
\lhead{\hmwkAuthorName}
\chead{\hmwkClassShort\ \hmwkTitle}
\rhead{\firstxmark}
\lfoot{\lastxmark}
\cfoot{\thepage}

\renewcommand\headrulewidth{0.4pt}
\renewcommand\footrulewidth{0.4pt}

\setlength\parindent{0pt}

%
% Create Problem Sections
%

\newcommand{\enterProblemHeader}[1]{
    \nobreak\extramarks{}{Problem {#1} continued on next page\ldots}\nobreak{}
    \nobreak\extramarks{{#1} (continued)}{{#1} continued on next page\ldots}\nobreak{}
}

\newcommand{\exitProblemHeader}[1]{
    \nobreak\extramarks{{#1} (continued)}{{#1} continued on next page\ldots}\nobreak{}
    % \stepcounter{#1}
    \nobreak\extramarks{{#1}}{}\nobreak{}
}

\setcounter{secnumdepth}{0}
\newcounter{partCounter}

\newcommand{\problemNumber}{0.0}

\newenvironment{homeworkProblem}[1][-1]{
    \renewcommand{\problemNumber}{{#1}}
    \section{\problemNumber}
    \setcounter{partCounter}{1}
    \enterProblemHeader{\problemNumber}
}{
    \exitProblemHeader{\problemNumber}
}

%
% Homework Details
%   - Title
%   - Class
%   - Author
%

\newcommand{\hmwkTitle}{Week\ \#7 Assignment}
\newcommand{\hmwkClassShort}{RBE 595}
\newcommand{\hmwkClass}{RBE 595 --- Reinforcement Learning}
\newcommand{\hmwkAuthorName}{\textbf{Arjan Gupta}}

%
% Title Page
%

\title{
    \vspace{2in}
    \textmd{\textbf{\hmwkClass}}\\
    \textmd{\textbf{\hmwkTitle}}\\
    \textmd{\textbf{Temporal Difference Learning}}\\
    \vspace{3in}
}

\author{\hmwkAuthorName}
\date{}

\renewcommand{\part}[1]{\textbf{\large Part \Alph{partCounter}}\stepcounter{partCounter}\\}

%
% Various Helper Commands
%

% Useful for algorithms
\newcommand{\alg}[1]{\textsc{\bfseries \footnotesize #1}}

% For derivatives
\newcommand{\deriv}[2]{\frac{\mathrm{d}}{\mathrm{d}#2} \left(#1\right)}

% For compact derivatives
\newcommand{\derivcomp}[2]{\frac{\mathrm{d}#1}{\mathrm{d}#2}}

% For partial derivatives
\newcommand{\pderiv}[2]{\frac{\partial}{\partial #2} \left(#1\right)}

% For compact partial derivatives
\newcommand{\pderivcomp}[2]{\frac{\partial #1}{\partial #2}}

% Integral dx
\newcommand{\dx}{\mathrm{d}x}

% Alias for the Solution section header
\newcommand{\solution}{\textbf{\large Solution}}

% Probability commands: Expectation, Variance, Covariance, Bias
\newcommand{\E}{\mathrm{E}}
\newcommand{\Var}{\mathrm{Var}}
\newcommand{\Cov}{\mathrm{Cov}}
\newcommand{\Bias}{\mathrm{Bias}}

\begin{document}

\maketitle

\nobreak\extramarks{Problem 1}{}\nobreak{}

\pagebreak

\begin{homeworkProblem}[Problem 1]
    Between DP (Dynamic Programming), MC (Monte-Carlo) and TD (Temporal Difference), which
    one of these algorithms use bootstrapping? Explain.

    \subsection{Answer}

    Bootstrapping is the process of updating the value of a state based on the value of a future state.
    
    \begin{itemize}
        \item \textbf{Dynamic Programming} (DP) uses bootstrapping. This is because DP uses the Bellman
        equation to update the value of a state based on the value of a future state.
        \item \textbf{Monte-Carlo} (MC) does not use bootstrapping. This is because MC does not use the
        Bellman equation to update the value of a state based on the value of a future state. Instead,
        MC uses the actual return value to update the value of a state.
        \item \textbf{Temporal Difference} (TD) uses bootstrapping. This is because TD uses the Bellman
        equation to update the value of a state based on the value of a future state.
    \end{itemize}


\end{homeworkProblem}

\nobreak\extramarks{Problem 2}{}\nobreak{}

\pagebreak

\begin{homeworkProblem}[Problem 2]
    We mentioned that the target value for TD is $[R_{t+1} + \gamma V(s_{t+1})]$. What is the target value for
    Monte-carlo, Q-learning, SARSA and Expected-SARSA?

    \subsection{Answer}

    \begin{itemize}
        \item \textbf{Monte-Carlo} (MC) does not use bootstrapping. Therefore, the target value is the
        actual return value, $G_t$.
        \item \textbf{Q-Learning} is an off-policy TD control algorithm. Therefore, the target value is
        $R_{t+1} + \gamma \max_{a} Q(S_{t+1}, a)$.
        \item \textbf{SARSA} is an on-policy TD control algorithm. Therefore, the target value is
        $R_{t+1} + \gamma Q(S_{t+1}, A_{t+1})$.
        \item \textbf{Expected-SARSA} is an on-policy TD control algorithm. Therefore, the target value is
        $R_{t+1} + \gamma \mathbb{E}_{\pi} \left[ Q(S_{t+1}, A_{t+1}) \mid S_{t+1} \right]$.
    \end{itemize}

\end{homeworkProblem}

\nobreak\extramarks{Problem 2}{}\nobreak{}

\pagebreak

\nobreak\extramarks{Problem 3}{}\nobreak{}

\begin{homeworkProblem}[Problem 3]
    What are the similarities of TD and MC?

    \subsection{Answer}

    The similarities between TD and MC are as follows:

    \begin{itemize}
        \item Both TD and MC are model-free.
        \item Both TD and MC are used for prediction and control.
    \end{itemize}
\end{homeworkProblem}

\pagebreak

\nobreak\extramarks{Problem 4}{}\nobreak{}

\begin{homeworkProblem}[Problem 4]
    Assume that we have two states $x$ and $y$ with the current value of $V(x) = 10$, $V(y) = 1$. We
    run an episode of $\{x, 3, y, 0, y, 5, T\}$. What's the new estimate of $V(x)$, $V(y)$ using TD (assume
    step size $\alpha = 0.1$ and discount rate $\gamma = 0.9$).

    \subsection{Answer}

    The new estimate of $V(x)$ is as follows:

    \begin{align*}
        V(x) &= V(x) + \alpha \left[ R_{t+1} + \gamma V(S_{t+1}) - V(x) \right]\\
             &= 10 + 0.1 \left[ 3 + 0.9 \cdot 1 - 10 \right]\\
             &= 10 + 0.1 \left[ 3.9 - 10 \right]\\
             &= 10 + 0.1 \left[ -6.1 \right]\\
             &= 10 - 0.61\\
             &= 9.39
    \end{align*}

    However, $V(y)$ gets updated twice in this episode. The first update is as follows:

    \begin{align*}
        V(y) &= V(y) + \alpha \left[ R_{t+1} + \gamma V(S_{t+1}) - V(y) \right]\\
             &= 1 + 0.1 \left[ 0 + 0.9 \cdot 1 - 1 \right]\\
                &= 1 + 0.1 \left[ 0.9 - 1 \right]\\
                &= 1 + 0.1 \left[ -0.1 \right]\\
                &= 1 - 0.01\\
                &= 0.99
    \end{align*}

    The second update is as follows:

    \begin{align*}
        V(y) &= V(y) + \alpha \left[ R_{t+1} + \gamma V(S_{t+1}) - V(y) \right]\\
             &= 0.99 + 0.1 \left[ 5 + 0.9 \cdot 0 - 0.99 \right]\\
                &= 0.99 + 0.1 \left[ 5 - 0.99 \right]\\
                &= 0.99 + 0.1 \left[ 4.01 \right]\\
                &= 0.99 + 0.401\\
                &= 1.391
    \end{align*}

    Therefore, the new estimate of $V(x)$ is $9.39$ and the new estimate of $V(y)$ is $1.391$.

\end{homeworkProblem}

\pagebreak

\nobreak\extramarks{Problem 5}{}\nobreak{}

\begin{homeworkProblem}[Problem 5]
    Can we consider TD an online (real-time) method and MC an offline method? Why?

    \subsection{Answer}
    
    Yes, we can consider TD an online (real-time) method and MC an offline method. This is because
    TD learns during the episode, whereas MC learns after the episode has ended. TD updates the
    value of a state based on the value of the next state (during the episode), whereas MC updates
    the value of a state based on the actual return value (after the entire episode has ended).
\end{homeworkProblem}

\pagebreak

\nobreak\extramarks{Problem 6}{}\nobreak{}

\begin{homeworkProblem}[Problem 6]

    Does Q-learning learn the outcome of exploratory actions? (Refer to the Cliff walking example).

    \subsection{Answer}

    Yes, Q-learning learns the outcome of exploratory actions. This is because Q-learning is an
    off-policy TD control algorithm. Therefore, Q-learning learns the optimal policy, $\pi_*$, which
    is the policy that maximizes the value function, $q_*$, i.e., $\pi_* = \arg\max_{\pi} q_*(s, a)$.
    This means that Q-learning learns the optimal policy, $\pi_*$, even if the behavior policy, $b$,
    is exploratory.

\end{homeworkProblem}

\pagebreak

\nobreak\extramarks{Problem 7}{}\nobreak{}

\begin{homeworkProblem}[Problem 7]

    \textbf{(Exercise 3.17)}
    What is the Bellman equation for action values, that
    is, for $q_{\pi}$? It must give the action value $q_{\pi}(s, a)$ in terms of the action
    values, $q_{\pi}(s', a')$, of possible successors to the state-action pair $(s, a)$.\\
    Hint: the backup diagram below corresponds to this equation.
    Show the sequence of equations analogous to (3.14), but for action
    values.

    \subsection{Answer}

    From the textbook, the action-value function for a policy $\pi$ is defined as,

    \begin{align*}
        q_{\pi}(s, a) &\doteq \mathbb{E}_{\pi} \left[ G_t \mid S_t = s, A_t = a \right]\\
                   &= \mathbb{E}_{\pi} \left[ \sum_{k=0}^{\infty} \gamma^k R_{t+k+1} \mid S_t = s, A_t = a \right]\\
                   &= \mathbb{E}_{\pi} \left[ R_{t+1} + \gamma \sum_{k=0}^{\infty} \gamma^k R_{t+k+2} \mid S_t = s, A_t = a \right]\\
                   &= \mathbb{E}_{\pi} \left[ R_{t+1} + \gamma G_{t+1} \mid S_t = s, A_t = a \right]\\
                     &= \mathbb{E}_{\pi} \left[ R_{t+1} \mid S_t = s, A_t = a \right] + \gamma \mathbb{E}_{\pi} \left[ G_{t+1} \mid S_t = s, A_t = a \right]\\
    \end{align*}

    Now, let us consider the first and second terms of the above equation separately.\\

    \textbf{First Term}\\
    \vspace{-0.25cm}
    \begin{align*}
        \mathbb{E}_{\pi} \left[ R_{t+1} \mid S_t = s, A_t = a \right] &= \sum_{r \in \mathcal{R}} r \cdot p(r \mid s, a)
        = \sum_{r \in \mathcal{R}} \sum_{s' \in \mathcal{S}} r \cdot p(s', r \mid s, a)
    \end{align*}

    \textbf{Second Term}\\
    \vspace{-0.25cm}
    \begin{align*}
        \gamma \mathbb{E}_{\pi} \left[ G_{t+1} \mid S_t = s, A_t = a \right] &= \gamma \sum_{g \in \mathcal{G}} g \cdot p(g \mid s, a)\\
        &= \gamma \sum_{g \in \mathcal{G}} \sum_{r \in \mathcal{R}} \sum_{s' \in \mathcal{S}} \sum_{a' \in \mathcal{A}} g \cdot p(g \mid s', a') \cdot p(s', r \mid s, a) \cdot \pi(a' \mid s')\\
    \end{align*}
    Where, $\sum_{g \in \mathcal{G}} g \cdot p(g \mid s', a') = \mathbb{E}_{\pi} \left[ G_{t+1} \mid S_{t+1} = s', A_{t+1} = a' \right] = q_{\pi}(s', a')$\\

    Therefore the second term is,

    \begin{align*}
        \gamma \mathbb{E}_{\pi} \left[ G_{t+1} \mid S_t = s, A_t = a \right] &= \gamma\sum_{r \in \mathcal{R}} \sum_{s' \in \mathcal{S}} \sum_{a' \in \mathcal{A}} q_{\pi}(s', a') \cdot p(s', r \mid s, a) \cdot \pi(a' \mid s')\\
    \end{align*}

    Now, combining the first and second terms, we get,

    \begin{align*}
        q_{\pi}(s, a) &=  \sum_{r \in \mathcal{R}} \sum_{s' \in \mathcal{S}} r \cdot p(s', r \mid s, a) + \gamma\sum_{r \in \mathcal{R}} \sum_{s' \in \mathcal{S}} \sum_{a' \in \mathcal{A}} q_{\pi}(s', a') \cdot p(s', r \mid s, a) \cdot \pi(a' \mid s')\\
        q_{\pi}(s, a) &= \sum_{s',r} p(s', r \mid s, a) \left[ r + \gamma\sum_{a'} \pi(a' \mid s') q_{\pi}(s', a') \right]\\
    \end{align*}

    Which is the Bellman equation for action values, i.e., for $q_{\pi}$.

    \subsubsection{Backup Diagram Confirmation}

    This equation can be verified by looking at the backup diagram given in the prompt. The backup diagram
    shows that we start with the state-action pair $(s, a)$. To get to the next state, we are subjected
    to the environment $p(s', r \mid s, a)$. The reward $r$ is added to the discounted return $G_{t+1}$.
    This brings us to our new state, $s'$. At this point, the equation would look as follows,

    \begin{align*}
        q_{\pi}(s, a) &= \sum_{s',r} p(s', r \mid s, a) \left[ r + \gamma v_{\pi}(s') \right]\\
    \end{align*}

    However we still need to eliminate the $v_{\pi}(s')$ term. To do this, we go through our
    policy, $\pi$, to get the action $a'$ that we would take in the state $s'$. Now the equation
    becomes,

    \begin{align*}
        q_{\pi}(s, a) &= \sum_{s',r} p(s', r \mid s, a) \left[ r + \gamma\sum_{a'} \pi(a'\mid s') q_{\pi}(s', a') \right]\\
    \end{align*}

    So, the Bellman equation for action values, i.e., for $q_{\pi}$, is confirmed by the backup diagram.

\end{homeworkProblem}

\pagebreak

\nobreak\extramarks{Problem 8}{}\nobreak{}

\begin{homeworkProblem}[Problem 8]

    \textbf{(Exercise 3.22)}
    Consider the continuing MDP shown below. The only decision to be made is that in the top state,
    where two actions are available, left and right. The numbers
    show the rewards that are received deterministically after
    each action. There are exactly two deterministic policies,
    $\pi_{\text{left}}$ and $\pi_{\text{right}}$.
    What policy is optimal if $\gamma = 0$? If $\gamma = 0.9$?
    If  $\gamma = 0.5$?

    \subsection{Answer}

    The discounted return is defined as,

    \[
    \tag{3.8}
        G_t \doteq R_{t+1} + \gamma R_{t+2} + \gamma^2 R_{t+3} + \cdots = \sum_{k=0}^{\infty} \gamma^k R_{t+k+1}
    \]

    \subsubsection{Case 1: $\gamma = 0$}

    When $\gamma = 0$, the left policy rewards are calculated as follows,

    \begin{align*}
        G_{\text{left}} &= 1 + 0 + 0 + \cdots = 1
    \end{align*}

    Similarly, the right policy rewards are calculated as follows,

    \begin{align*}
        G_{\text{right}} &= 0 + 0 + \cdots = 0
    \end{align*}

    In this case, the \textbf{left} policy is optimal.

    \subsubsection{Case 2: $\gamma = 0.9$}

    When $\gamma = 0.9$, the left policy rewards are calculated as follows,

    \begin{align*}
        G_{\text{left}} &= 1 + 0.9 \cdot 0 + 0.9^2 \cdot 1 + \cdots\\
                        &= 1 + 0.9^2 + 0.9^4 + \cdots\\
                        &= \sum_{k=0}^{\infty} 0.9^{2k}\\
                        &= \sum_{k=0}^{\infty} 0.81^{k}\\
                        &= \frac{1}{1 - 0.81}
                        = \frac{1}{0.19}\\
                        &= 5.263
    \end{align*}

    Similarly, the right policy rewards are calculated as follows,

    \begin{align*}
        G_{\text{right}} &= 0 + 0.9 \cdot 2 + 0 + 0.9^3 \cdot 2 + \cdots \\
                        &= 0.9 \cdot 2 + 0.9^3 \cdot 2 + \cdots\\
                        &= 2 \cdot \sum_{k=0}^{\infty} 0.9^{2k+1}
                        = 2 \cdot \sum_{k=0}^{\infty} (0.9)(0.81)^{k}
                        = 2 \cdot \frac{0.9}{1 - 0.81}\\
                        &= \frac{1.8}{0.19} = 9.474
    \end{align*}

    In this case, the \textbf{right} policy is optimal.

    \subsubsection{Case 3: $\gamma = 0.5$}

    When $\gamma = 0.5$, the left policy rewards are calculated as follows,

    \begin{align*}
        G_{\text{left}} &= 1 + 0.5 \cdot 0 + 0.5^2 \cdot 1 + \cdots\\
                        &= 1 + 0.5^2 + 0.5^4 + \cdots\\
                        &= \sum_{k=0}^{\infty} 0.5^{2k}
                        = \sum_{k=0}^{\infty} 0.25^{k}\\
                        &= \frac{1}{1 - 0.25}
                        = \frac{1}{0.75}\\
                        &= 1.333
    \end{align*}

    Similarly, the right policy rewards are calculated as follows,

    \begin{align*}
        G_{\text{right}} &= 0 + 0.5 \cdot 2 + 0 + 0.5^3 \cdot 2 + \cdots \\
                        &= 0.5 \cdot 2 + 0.5^3 \cdot 2 + \cdots\\
                        &= 2 \cdot \sum_{k=0}^{\infty} 0.5^{2k+1}
                        = 2 \cdot \sum_{k=0}^{\infty} (0.5)(0.25)^{k}
                        = 2 \cdot \frac{0.5}{1 - 0.25}\\
                        &= \frac{1}{0.75} = 1.333
    \end{align*}

    In this case, both the \textbf{left} and \textbf{right} policies are optimal.

\end{homeworkProblem}

\end{document}