\documentclass{article}

\usepackage{fancyhdr}
\usepackage{extramarks}
\usepackage{amsmath}
\usepackage{amsthm}
\usepackage{amsfonts}
\usepackage{tikz}
\usepackage[plain]{algorithm}
\usepackage{algpseudocode}
\usepackage{graphicx}
\usepackage{gensymb}
\usepackage{hyperref}

\DeclareRobustCommand{\bbone}{\text{\usefont{U}{bbold}{m}{n}1}}

\DeclareMathOperator{\EX}{\mathbb{E}}% expected value

\graphicspath{{./images/}}

\usetikzlibrary{automata,positioning}

%
% Basic Document Settings
%

\topmargin=-0.45in
\evensidemargin=0in
\oddsidemargin=0in
\textwidth=6.5in
\textheight=9.0in
\headsep=0.25in

\linespread{1.1}

\pagestyle{fancy}
\lhead{\hmwkAuthorName}
\chead{\hmwkClassShort\ \hmwkTitle}
\rhead{\firstxmark}
\lfoot{\lastxmark}
\cfoot{\thepage}

\renewcommand\headrulewidth{0.4pt}
\renewcommand\footrulewidth{0.4pt}

\setlength\parindent{0pt}

%
% Create Problem Sections
%

\newcommand{\enterProblemHeader}[1]{
    \nobreak\extramarks{}{Problem {#1} continued on next page\ldots}\nobreak{}
    \nobreak\extramarks{{#1} (continued)}{{#1} continued on next page\ldots}\nobreak{}
}

\newcommand{\exitProblemHeader}[1]{
    \nobreak\extramarks{{#1} (continued)}{{#1} continued on next page\ldots}\nobreak{}
    % \stepcounter{#1}
    \nobreak\extramarks{{#1}}{}\nobreak{}
}

\setcounter{secnumdepth}{0}
\newcounter{partCounter}

\newcommand{\problemNumber}{0.0}

\newenvironment{homeworkProblem}[1][-1]{
    \renewcommand{\problemNumber}{{#1}}
    \section{\problemNumber}
    \setcounter{partCounter}{1}
    \enterProblemHeader{\problemNumber}
}{
    \exitProblemHeader{\problemNumber}
}

%
% Homework Details
%   - Title
%   - Class
%   - Author
%

\newcommand{\hmwkTitle}{Assignment\ \#6}
\newcommand{\hmwkClassShort}{RBE 595}
\newcommand{\hmwkClass}{RBE 595 --- Reinforcement Learning}
\newcommand{\hmwkAuthorName}{\textbf{Arjan Gupta}}

%
% Title Page
%

\title{
    \vspace{2in}
    \textmd{\textbf{\hmwkClass}}\\
    \textmd{\textbf{\hmwkTitle}}\\
    \textmd{\textbf{Model-Based Reinforcement Learning}}\\
    \vspace{3in}
}

\author{\hmwkAuthorName}
\date{}

\renewcommand{\part}[1]{\textbf{\large Part \Alph{partCounter}}\stepcounter{partCounter}\\}

%
% Various Helper Commands
%

% Useful for algorithms
\newcommand{\alg}[1]{\textsc{\bfseries \footnotesize #1}}

% For derivatives
\newcommand{\deriv}[2]{\frac{\mathrm{d}}{\mathrm{d}#2} \left(#1\right)}

% For compact derivatives
\newcommand{\derivcomp}[2]{\frac{\mathrm{d}#1}{\mathrm{d}#2}}

% For partial derivatives
\newcommand{\pderiv}[2]{\frac{\partial}{\partial #2} \left(#1\right)}

% For compact partial derivatives
\newcommand{\pderivcomp}[2]{\frac{\partial #1}{\partial #2}}

% Integral dx
\newcommand{\dx}{\mathrm{d}x}

% Alias for the Solution section header
\newcommand{\solution}{\textbf{\large Solution}}

% Probability commands: Expectation, Variance, Covariance, Bias
\newcommand{\E}{\mathrm{E}}
\newcommand{\Var}{\mathrm{Var}}
\newcommand{\Cov}{\mathrm{Cov}}
\newcommand{\Bias}{\mathrm{Bias}}

\begin{document}

\maketitle

\nobreak\extramarks{Problem 1}{}\nobreak{}

\pagebreak

\begin{homeworkProblem}[Problem 1]
    What is ``planning'' in the context of Reinforcement Learning?

    \subsection{Answer}

    In the context of Reinforcement Learning, planning is the process of using a model of the
    environment to improve the policy.\\

    A model of the environment is a representation of the environment that is used to predict
    what next state and reward will be given a current state and action. Once we have a model,
    we can use it to simulate the environment and produce a simulated experience, in the
    form of an episode. Then we can use this simulated experience to improve the policy. This
    is the process of \textbf{planning}.\\ 

\end{homeworkProblem}

\nobreak\extramarks{Problem 2}{}\nobreak{}

\pagebreak

\begin{homeworkProblem}[Problem 2]
    What is the purpose of introducing Control Variates in per-decision importance sampling?

    \subsection{Answer}

    The purpose of introducing Control Variates in per-decision importance sampling is
    to further reduce the variance of the estimate of the return.\\
    
    Plain per-decision importance
    sampling reduces the variance of the estimate of the return by making sure that the estimate
    of the whole return is not 0 every time the behavior policy takes an action
    that the target policy would not have taken. However, the variance of the estimate of the
    return can be decreased even further by using Control Variates. This is done
    by using a `control variate term' in the equation for $G_{t:h}$. Without
    control variates, the equation for $G_{t:h}$ is as follows:

    \begin{align*}
        G_{t:h} = R_{t+1} + \gamma G_{t+1:h}, \quad t < h < T
    \end{align*}

    With control variates, the equation for $G_{t:h}$ is as follows:

    \begin{align*}
        G_{t:h} \doteq \rho_{t} \left[ R_{t+1} + \gamma G_{t+1:h}] + (1 - \rho_{t}) V_{h-1}(S_t) \right], \quad t < h < T
    \end{align*}

    Where the control variate term is $(1 - \rho_{t}) V_{h-1}(S_t)$.\\

    This equation uses a `convex combination' of the estimate of the two terms. This way, 
    when $\rho_{t} = 0$, we have $G_{t:h} = V_{h-1}(S_t)$, which is the current
    estimate of the return for the current state. Overall this reduces the `jaggedness' in the
    plot of the estimate of the return over time, which reduces the variance of the estimate
    of the return.\\

    It is also possible to prove that the control variate term does not add bias
    to the estimate of the return.

    \subsubsection{Control variates for action-value estimation}

    With control-estimates, the equation for $G_{t:h}$ in action-value estimation is as follows:

    \begin{align*}
        G_{t:h} = R_{t+1} + \gamma \rho_{t+1} (G_{t+1:h} - Q(S_{t+1}, A_{t+1})) + \gamma V_{h-1}(S_{t+1})
    \end{align*}

    Where $V(S) = \sum_{a} \pi(a|S) Q(S, a)$.\\

    Here, when $\rho_{t} = 0$, we have $G_{t:h} = R_{t+1} + \gamma V_{h-1}(S_{t+1})$, instead
    of simply $G_{t:h} = R_{t+1}$, which would be the case without control variates. This also
    reduces the variance of the estimate of the return.\\

\end{homeworkProblem}

\nobreak\extramarks{Problem 2}{}\nobreak{}

\pagebreak

\nobreak\extramarks{Problem 3}{}\nobreak{}

\begin{homeworkProblem}[Problem 3]
    In off-policy learning, what are the pros and cons of the Tree-Backup algorithm versus off-policy
    SARSA (comment on the complexity, exploration, variance, and bias, and others)?

    \subsection{Answer}

    The pros and cons of the Tree-Backup algorithm versus off-policy SARSA are as follows:

    \begin{itemize}
        \item \textbf{Complexity:} The computational complexity of both the Tree-Backup algorithm and
            off-policy SARSA over a single episode is $O(n^2)$, where $n$ is the number of steps 
            in the episode. This is because the outer loop of both algorithms iterates over the
            steps in the episode, and the inner loop of both algorithms are used to calculate iterative
            sums for the estimate of the return. In the case of off-policy SARSA, one of the inner loops is
            also used to calculate the importance sampling ratio. For $k$ episodes, the complexity of
            both algorithms is $O(kn^2)$. Therefore, from a complexity standpoint, neither algorithm
            is better than the other.
        \item \textbf{Exploration:} Both the Tree-Backup algorithm and off-policy SARSA are off-policy
            algorithms. Therefore, both algorithms can be used to explore the environment. 
        \item \textbf{Variance:} The variance of the Tree-Backup algorithm is lower than that of 
        off-policy SARSA\@. This is because the off-policy SARSA algorithm uses the importance sampling
        ratio, which can cause the variance of the estimate of the return to increase, especially
        when control variates are not used. The Tree-Backup algorithm does not use the importance
        sampling ratio, and therefore does not have this problem.
        \item \textbf{Bias:} As a trade-off for lower variance, the Tree-Backup algorithm has
            higher bias than off-policy SARSA\@.
        \item \textbf{Others:} One benefit of using Tree-Backup algorithm over off-policy SARSA is
        that the Tree-Backup algorithm can be used when we have no knowledge of the underlying
        distribution of the behavior policy. This can be useful depending on the application, for
        example, if the behavior policy is a human, and we have no knowledge of the human's
        decision-making process.
    \end{itemize}
\end{homeworkProblem}

\pagebreak

\nobreak\extramarks{Problem 4}{}\nobreak{}

\begin{homeworkProblem}[Problem 4]
    \textbf{(Exercise 7.4)} Prove that the $n$-step return of Sarsa (7.4) can be written 
    exactly in terms of a novel TD error, as

    \begin{align}\tag{7.6}
        G_{t:t+n} = Q_{t-1}(S_t, A_t) + \sum_{k=t}^{min(t+n, T)-1} \gamma^{k-t} [R_{k+1} + \gamma Q_k(S_{k+1}, A_{k+1}) - Q_{k-1}(S_k, A_k)]
    \end{align}

    \subsection{Answer}

    The $n$-step return of Sarsa (7.4) is as follows:

    \begin{align*}
        G_{t:t+n} = R_{t+1} + \gamma R_{t+2} + \ldots + \gamma^{n-1} R_{t+n} + \gamma^n Q_{t+n-1}(S_{t+n}, A_{t+n}), \quad n \geq 1, \quad  0 \leq t < T-n
    \end{align*}

    We can rewrite this as follows:

    \begin{align}\tag{Equation 1, Problem 4}
        G_{t:t+n} &= \sum_{i=1}^{n} \gamma^{i-1} R_{t+i} + \gamma^n Q_{t+n-1}(S_{t+n}, A_{t+n})
    \end{align}

    Now we take two cases of equation 7.6 --- one where $t+n < T$, and one where $t+n \geq T$.

    \subsubsection{Case 1: $t+n < T$}

    In this case, we have $min(t+n, T) = t+n$. Therefore, equation 7.6 becomes:

    \begin{align*}
        G_{t:t+n} &= Q_{t-1}(S_t, A_t) + \sum_{k=t}^{t+n-1} \gamma^{k-t} [R_{k+1} + \gamma Q_k(S_{k+1}, A_{k+1}) - Q_{k-1}(S_k, A_k)]
    \end{align*}

    Let us expand the summation in the above equation:

    \begin{align*}
        G_{t:t+n} &= Q_{t-1}(S_t, A_t) + \gamma^{0} [R_{t+1} + \gamma Q_t(S_{t+1}, A_{t+1}) - Q_{t-1}(S_t, A_t)]\\
                  &+ \gamma^{1} [R_{t+2} + \gamma Q_{t+1}(S_{t+2}, A_{t+2}) - Q_{t}(S_{t+1}, A_{t+1})]\\
                  &+ \gamma^{2} [R_{t+3} + \gamma Q_{t+2}(S_{t+3}, A_{t+3}) - Q_{t+1}(S_{t+2}, A_{t+2})]\\
                  &+ \gamma^{3} [R_{t+4} + \gamma Q_{t+3}(S_{t+4}, A_{t+4}) - Q_{t+2}(S_{t+3}, A_{t+3})]\\
                  &+ \ldots\\
                  &+ \gamma^{n-1} [R_{t+n} + \gamma Q_{t+n-1}(S_{t+n}, A_{t+n}) - Q_{t+n-2}(S_{t+n-1}, A_{t+n-1})]\\
                  &= Q_{t-1}(S_t, A_t)\\
                  &+ R_{t+1} + \gamma Q_t(S_{t+1}, A_{t+1}) - Q_{t-1}(S_t, A_t)\\
                &+ \gamma R_{t+2} + \gamma^2 Q_{t+1}(S_{t+2}, A_{t+2}) - \gamma Q_{t}(S_{t+1}, A_{t+1})\\
                &+ \gamma^2 R_{t+3} + \gamma^3 Q_{t+2}(S_{t+3}, A_{t+3}) - \gamma^2 Q_{t+1}(S_{t+2}, A_{t+2})\\
                &+ \gamma^3 R_{t+4} + \gamma^4 Q_{t+3}(S_{t+4}, A_{t+4}) - \gamma^3 Q_{t+2}(S_{t+3}, A_{t+3})\\
                &+ \ldots\\
                &+ \gamma^{n-1} R_{t+n} + \gamma^n Q_{t+n-1}(S_{t+n}, A_{t+n}) - \gamma^{n-1} Q_{t+n-2}(S_{t+n-1}, A_{t+n-1})\\
    \end{align*}

    However, we can see that the terms in the above equation cancel out. The cancellation
    pattern is, $Q_{t-1}(S_t, A_t)$ from the first line gets cancelled by the last term in the second line,
    $\gamma Q_{t}(S_{t+1}, A_{t+1})$ gets cancelled by the last term in the third line, and so on.\\

    Therefore, we are left with the following:

    \begin{align*}
        G_{t:t+n} &= R_{t+1} + \gamma R_{t+2} + \ldots + \gamma^{n-1} R_{t+n} + \gamma^n Q_{t+n-1}(S_{t+n}, A_{t+n})\\
                  &= \sum_{i=1}^{n} \gamma^{i-1} R_{t+i} + \gamma^n Q_{t+n-1}(S_{t+n}, A_{t+n})
    \end{align*}

    Which is the same as the rewritten form of the $n$-step return of Sarsa (7.4), as shown in
    equation 1 of this Problem. Therefore, this case is proven.\\

    \subsubsection{Case 2: $t+n \geq T$}

    In this case, we have $min(t+n, T) = T$. Let us assume that $t+n$ overshoots $T$ by $x$ steps,
    where $x \geq 0$. So, $t+n = T + x$, or $T = t+n-x$. Therefore, equation 7.6 becomes:

    \begin{align*}
        G_{t:t+n} &= Q_{t-1}(S_t, A_t) + \sum_{k=t}^{t+n-x-1} \gamma^{k-t} [R_{k+1} + \gamma Q_k(S_{k+1}, A_{k+1}) - Q_{k-1}(S_k, A_k)]
    \end{align*}

    Let us expand the summation in the above equation:

    \begin{align*}
        G_{t:t+n} &= Q_{t-1}(S_t, A_t) + \gamma^{0} [R_{t+1} + \gamma Q_t(S_{t+1}, A_{t+1}) - Q_{t-1}(S_t, A_t)]\\
                  &+ \gamma^{1} [R_{t+2} + \gamma Q_{t+1}(S_{t+2}, A_{t+2}) - Q_{t}(S_{t+1}, A_{t+1})]\\
                  &+ \gamma^{2} [R_{t+3} + \gamma Q_{t+2}(S_{t+3}, A_{t+3}) - Q_{t+1}(S_{t+2}, A_{t+2})]\\
                  &+ \gamma^{3} [R_{t+4} + \gamma Q_{t+3}(S_{t+4}, A_{t+4}) - Q_{t+2}(S_{t+3}, A_{t+3})]\\
                  &+ \ldots\\
                  &+ \gamma^{n-x-1} [R_{t+n-x} + \gamma Q_{t+n-x-1}(S_{t+n-x}, A_{t+n-x}) - Q_{t+n-x-2}(S_{t+n-x-1}, A_{t+n-x-1})]\\
                  &= Q_{t-1}(S_t, A_t)\\
                  &+ R_{t+1} + \gamma Q_t(S_{t+1}, A_{t+1}) - Q_{t-1}(S_t, A_t)\\
                &+ \gamma R_{t+2} + \gamma^2 Q_{t+1}(S_{t+2}, A_{t+2}) - \gamma Q_{t}(S_{t+1}, A_{t+1})\\
                &+ \gamma^2 R_{t+3} + \gamma^3 Q_{t+2}(S_{t+3}, A_{t+3}) - \gamma^2 Q_{t+1}(S_{t+2}, A_{t+2})\\
                &+ \gamma^3 R_{t+4} + \gamma^4 Q_{t+3}(S_{t+4}, A_{t+4}) - \gamma^3 Q_{t+2}(S_{t+3}, A_{t+3})\\
                &+ \ldots\\
                &+ \gamma^{n-x-1} R_{t+n-x} + \gamma^{n-x} Q_{t+n-x-1}(S_{t+n-x}, A_{t+n-x}) - \gamma^{n-x-1} Q_{t+n-x-2}(S_{t+n-x-1}, A_{t+n-x-1})\\
    \end{align*}

    However, we can see that the terms in the above equation cancel out in the same way as in Case 1.

    \begin{align*}
        G_{t:t+n} &= R_{t+1} + \gamma R_{t+2} + \ldots + \gamma^{n-x-1} R_{t+n-x} + \gamma^{n-x} Q_{t+n-x-1}(S_{t+n-x}, A_{t+n-x})\\
                  &= \sum_{i=1}^{n-x} \gamma^{i-1} R_{t+i} + \gamma^{n-x} Q_{t+n-x-1}(S_{t+n-x}, A_{t+n-x})
    \end{align*}

    And as we assumed, $x$ is the number of steps by which $t+n$ overshoots $T$. Therefore, $n-x$ is
    the `reduced horizon' of the $n$-step return of Sarsa (7.4). So, in general, as the horizon
    shrinks, we can keep taking $n-x = n$ steps, and thus the equation above can simply be written
    generally as:

    \begin{align*}
        G_{t:t+n} &= \sum_{i=1}^{n} \gamma^{i-1} R_{t+i} + \gamma^{n} Q_{t+n-1}(S_{t+n}, A_{t+n})
    \end{align*}

    Which is the same as the rewritten form of the $n$-step return of Sarsa (7.4), as shown in
    equation 1 of this Problem. Therefore, this case is also proven.\\

\end{homeworkProblem}

\end{document}