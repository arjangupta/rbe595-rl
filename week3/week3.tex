\documentclass{article}

\usepackage{fancyhdr}
\usepackage{extramarks}
\usepackage{amsmath}
\usepackage{amsthm}
\usepackage{amsfonts}
\usepackage{tikz}
\usepackage[plain]{algorithm}
\usepackage{algpseudocode}
\usepackage{graphicx}
\usepackage{gensymb}
\usepackage[framed,numbered,autolinebreaks,useliterate]{mcode}
\usepackage{listings}
\usepackage{hyperref}

\graphicspath{{./images/}}

\usetikzlibrary{automata,positioning}

%
% Basic Document Settings
%

\topmargin=-0.45in
\evensidemargin=0in
\oddsidemargin=0in
\textwidth=6.5in
\textheight=9.0in
\headsep=0.25in

\linespread{1.1}

\pagestyle{fancy}
\lhead{\hmwkAuthorName}
\chead{\hmwkClassShort\ \hmwkTitle}
\rhead{\firstxmark}
\lfoot{\lastxmark}
\cfoot{\thepage}

\renewcommand\headrulewidth{0.4pt}
\renewcommand\footrulewidth{0.4pt}

\setlength\parindent{0pt}

%
% Create Problem Sections
%

\newcommand{\enterProblemHeader}[1]{
    \nobreak\extramarks{}{Problem {#1} continued on next page\ldots}\nobreak{}
    \nobreak\extramarks{{#1} (continued)}{{#1} continued on next page\ldots}\nobreak{}
}

\newcommand{\exitProblemHeader}[1]{
    \nobreak\extramarks{{#1} (continued)}{{#1} continued on next page\ldots}\nobreak{}
    % \stepcounter{#1}
    \nobreak\extramarks{{#1}}{}\nobreak{}
}

\setcounter{secnumdepth}{0}
\newcounter{partCounter}

\newcommand{\problemNumber}{0.0}

\newenvironment{homeworkProblem}[1][-1]{
    \renewcommand{\problemNumber}{{#1}}
    \section{\problemNumber}
    \setcounter{partCounter}{1}
    \enterProblemHeader{\problemNumber}
}{
    \exitProblemHeader{\problemNumber}
}

%
% Homework Details
%   - Title
%   - Class
%   - Author
%

\newcommand{\hmwkTitle}{Week\ \#3 Assignment}
\newcommand{\hmwkClassShort}{RBE 595}
\newcommand{\hmwkClass}{RBE 595 --- Reinforcement Learning}
\newcommand{\hmwkAuthorName}{\textbf{Arjan Gupta}}

%
% Title Page
%

\title{
    \vspace{2in}
    \textmd{\textbf{\hmwkClass}}\\
    \textmd{\textbf{\hmwkTitle}}\\
    \vspace{3in}
}

\author{\hmwkAuthorName}
\date{}

\renewcommand{\part}[1]{\textbf{\large Part \Alph{partCounter}}\stepcounter{partCounter}\\}

%
% Various Helper Commands
%

% Useful for algorithms
\newcommand{\alg}[1]{\textsc{\bfseries \footnotesize #1}}

% For derivatives
\newcommand{\deriv}[2]{\frac{\mathrm{d}}{\mathrm{d}#2} \left(#1\right)}

% For compact derivatives
\newcommand{\derivcomp}[2]{\frac{\mathrm{d}#1}{\mathrm{d}#2}}

% For partial derivatives
\newcommand{\pderiv}[2]{\frac{\partial}{\partial #2} \left(#1\right)}

% For compact partial derivatives
\newcommand{\pderivcomp}[2]{\frac{\partial #1}{\partial #2}}

% Integral dx
\newcommand{\dx}{\mathrm{d}x}

% Alias for the Solution section header
\newcommand{\solution}{\textbf{\large Solution}}

% Probability commands: Expectation, Variance, Covariance, Bias
\newcommand{\E}{\mathrm{E}}
\newcommand{\Var}{\mathrm{Var}}
\newcommand{\Cov}{\mathrm{Cov}}
\newcommand{\Bias}{\mathrm{Bias}}

\begin{document}

\maketitle

\nobreak\extramarks{Problem 1}{}\nobreak{}

\pagebreak

\begin{homeworkProblem}[Problem 1]
    Suppose $\gamma = 0.8$ and we get the following sequence of rewards\\
    \[R_1 = -2,\ R_2 = 1,\ R_3 = 3,\ R_4 = 4,\ R_5 = 1.0\]
    Calculate the value of $G_0$ by using the equation 3.8 (work forward) and 3.9 (work backward) and
    show they yield the same results.

    \subsection{Answer}

    \subsubsection{Work Forward}
    From the the book, the \textit{discounted return} (equation 3.8), $G_t$, is defined as,

    \[
    \tag{3.8}
        G_t \doteq R_{t+1} + \gamma R_{t+2} + \gamma^2 R_{t+3} + \ldots = \sum_{k=0}^{\infty} \gamma^k R_{t+k+1}
    \]

    Plugging in the values from this problem, we get,
    \vspace{-0.1cm}
    \begin{align*}
        G_0 &= R_1 + \gamma R_2 + \gamma^2 R_3 + \gamma^3 R_4 + \gamma^4 R_5\\
        &= -2 + 0.8 \cdot 1 + 0.8^2 \cdot 3 + 0.8^3 \cdot 4 + 0.8^4 \cdot 1\\
        &= - 2 + 0.8 + 0.64 \cdot 3 + 0.512 \cdot 4 + 0.4096\\
        &= 3.1776
    \end{align*}

    \subsubsection{Work Backward}
    From the book, the ``recursive'' representation of \textit{discounted return} (equation 3.9), $G_t$, is defined as,

    \[
    \tag{3.9}
        G_t \doteq R_{t+1} + \gamma G_{t+1}
    \]

    Plugging in the values from this problem, we get,
    \vspace{-0.1cm}
    \begin{align*}
        G_0 &= R_1 + \gamma G_1\\
        &= -2 + 0.8 \cdot G_1
    \end{align*}
    \vspace{-0.3cm}
    Where we apply 3.8 to $G_1$,
    \vspace{-0.1cm}
    \begin{align*}
        G_1 &= R_2 + \gamma R_3 + \gamma^2 R_4 + \gamma^3 R_5\\
        &= 1 + 0.8 \cdot 3 + 0.8^2 \cdot 4 + 0.8^3 \cdot 1\\
        &= 6.472
    \end{align*}
    \vspace{-0.3cm}
    Therefore,
    \begin{align*}
        G_0 &= -2 + 0.8 \cdot G_1\\
        &= -2 + 0.8 \cdot 6.472\\
        &= 3.1776
    \end{align*}

    \subsubsection{Conclusion}
    We see that both methods yield the same result, $G_0 = 3.1776$.
\end{homeworkProblem}

\nobreak\extramarks{Problem 2}{}\nobreak{}

\pagebreak

\begin{homeworkProblem}[Problem 2]
    Explain how a room temperature control system can be modeled as an MDP? What are the
    states, actions, rewards, and transitions.

    \subsection{Answer}

    A room temperature control system can be modeled as an MDP as follows.\\

    \textbf{Scope}\\
    % \vspace{-0.25cm}\\

    Let us make some assumptions to define the scope of the solution.
    \begin{itemize}
        \item The temperatures are being measured in Fahrenheit.
        \item The temperature resolution of the temperature sensor in the room is $1\degree $F.
        \item Given the climate of the area, the room naturally stays between the range of $40\degree $F and $90\degree$F.
        \item The humans in the room are comfortable with temperatures between $68\degree $F and $72\degree$F.
    \end{itemize}
    \vspace{0.25cm}

    \textbf{States}\\
    % \vspace{-0.25cm}\\
    
    Therefore, the states of the system are the temperatures 
    in the room, $S = \{s \in \mathbb{Z} \mid 40 \leq s \leq 90\}$.\\

    \textbf{Actions}\\
    % \vspace{-0.25cm}\\

    The actions of the system are the temperature changes in the room. Assume that the control system
    can change the temperature by up to $5\degree $F in either direction. Therefore, in general,
    the set of all actions are
    $A = \{a \in \mathbb{Z} \mid -5 \leq a \leq 5\}$. However, the action at each state is limited by the
    state itself. For example, if the current temperature is below $68\degree $F, then the action
    cannot be to decrease the temperature further. Therefore, the set of actions can take on three
    possible sub-sets of $A$ depending on the state, as follows,
    \begin{itemize}
        \item $A_{\text{low}} = \{a \in A \mid a \geq 0\}$, if $s \leq 68$
        \item $A_{\text{mid}} = \{a \in A \mid -1 \leq a \leq 1\}$, if $68 < s < 72$
        \item $A_{\text{high}} = \{a \in A \mid a \leq 0\}$, if $s \geq 72$
    \end{itemize}
    \vspace{0.5cm}

    \textbf{Rewards}\\
    % \vspace{-0.25cm}\\
    
    The reward for the system is defined as the difference between the current temperature and the
    desired temperature. Therefore, the reward function is defined as,
    \[
        r(s,a,s') = \begin{cases}
            \lvert 70 - s \rvert, & \text{if } 68 \leq s \leq 72\\
            68 - s, & \text{if } s < 68\\
            s - 72, & \text{if } s > 72
        \end{cases}
    \]

    Notice that the reward is always non-negative. If the temperature does not change, then the reward
    is zero. If the temperature changes (the direction of which is enforced by the action set), then the
    reward is positive.\\

    % Define the transitions depending on the action taken in each state-level. Use \alpha with
    % subscript to represent the probability of the action being taken. For example, \alpha_{low}
    % represents the probability of the action being taken when the state is low. Use \alpha_{mid}
    % and \alpha_{high} for the other two states.
    \textbf{Transitions}\\
    % \vspace{-0.25cm}\\

    The transitions are defined as follows,

    \[
        p(s' \mid s, a) = \begin{cases}
            \alpha_{\text{low}}, & \text{if } s \leq 68 \text{ and } s' = s + a\\
            \alpha_{\text{mid}}, & \text{if } 68 < s < 72 \text{ and } s' = s + a\\
            \alpha_{\text{high}}, & \text{if } s \geq 72 \text{ and } s' = s + a\\
            1 - \alpha_{\text{low}}, & \text{if } s \leq 68 \text{ and } s' = s\\
            1 - \alpha_{\text{mid}}, & \text{if } 68 < s < 72 \text{ and } s' = s\\
            1 - \alpha_{\text{high}}, & \text{if } s \geq 72 \text{ and } s' = s\\
            0, & \text{otherwise}
        \end{cases}
    \]
    where $\alpha_{\text{low}}$, $\alpha_{\text{mid}}$, and $\alpha_{\text{high}}$ are the probabilities of
    the actions being taken when the state is low, mid, and high respectively. The value of these $\alpha$'s
    would vary depending on how effective the cooling and heating systems are. For example, if the cooling
    system is very effective, then $\alpha_{\text{low}}$ would be high. Similarly, if the heating system is
    very effective, then $\alpha_{\text{high}}$ would be high.\\

    % Form a table with columns s, a, s', p(s' | s, a), and r(s, a, s'). Fill in the table with
    % value ranges for s, a, and s'. Use the equations above to fill in the values for p(s' | s, a)
    % and r(s, a, s').
    \textbf{Tabular Summary}\\
    % \vspace{-0.25cm}\\

    The tabular summary of the MDP is as follows,

    \begin{center}
        \begin{tabular}{|c|c|c|c|c|}
            \hline
            $s$ & $a$ & $s'$ & $p(s' \mid s, a)$ & $r(s, a, s')$\\
            \hline
            $40 \leq s \leq 68$ & $a \geq 0$ & $s + a$ & $\alpha_{\text{low}}$ & $68 - s$\\
            \hline
            $40 \leq s \leq 68$ & $a \geq 0$ & $s$ & $1 - \alpha_{\text{low}}$ & $68 - s = 0$\\
            \hline
            $68 < s < 72$ & $-1 \leq a \leq 1$ & $s + a$ & $\alpha_{\text{mid}}$ & $\lvert 70 - s \rvert$\\
            \hline
            $68 < s < 72$ & $-1 \leq a \leq 1$ & $s$ & $1 - \alpha_{\text{mid}}$ & $\lvert 70 - s \rvert = 0$\\
            \hline
            $72 \leq s \leq 90$ & $a \leq 0$ & $s + a$ & $\alpha_{\text{high}}$ & $s - 72$\\
            \hline
            $72 \leq s \leq 90$ & $a \leq 0$ & $s$ & $1 - \alpha_{\text{high}}$ & $s - 72 = 0$\\
            \hline
        \end{tabular}
    \end{center}

\end{homeworkProblem}

\nobreak\extramarks{Problem 2}{}\nobreak{}

\pagebreak

\nobreak\extramarks{Problem 3}{}\nobreak{}

\begin{homeworkProblem}[Problem 3]
    What is the reward hypothesis in RL?

    \subsection{Answer}

    The textbook states the \textit{reward hypothesis} as follows,
    \begin{quote}
        ``That all of what we mean by goals and purposes can be well thought of as the maximization
        of the expected value of the cumulative sum of a received scalar signal (called reward).''
    \end{quote}

    Here is a simplified break-down of what the reward hypothesis means:
    \begin{itemize}
        \item In RL, we talk about goals and purposes, which is to find best way to solve a problem.
        \item Any solution to a complex problem can be broken down into a series of steps, and each step can have
        a value associated with it.
        \item We design this `value' associated with each step as a scalar signal which is received from the environment. This scalar signal is called the \textit{reward}.
        \item Therefore, \textbf{we hypothesize that} our all goals can be achieved by the maximization of the expected cumulative reward.
        \item A paper from 2021 titled ``Reward is enough'' by David Silver, Satinder Singh, Doina Precup, and Richard S. Sutton discusses this hypothesis in detail.
    \end{itemize}
\end{homeworkProblem}

\pagebreak

\nobreak\extramarks{Problem 4}{}\nobreak{}

\begin{homeworkProblem}[Problem 4]
    We have an agent in maze-like world. We want the agent to find the goal as soon as possible.
    We set the reward for reaching the goal equal to $+1$ With $\gamma = 1$. But we notice that the agent
    does not always reach the goal as soon as possible. How can we fix this?

    \subsection{Answer}

    TODO

\end{homeworkProblem}

\pagebreak

\nobreak\extramarks{Problem 5}{}\nobreak{}

\begin{homeworkProblem}[Problem 5]
    What is the difference between policy and action?

    \subsection{Answer}
    TODO
    
\end{homeworkProblem}

\pagebreak

\nobreak\extramarks{Problem 6}{}\nobreak{}

\begin{homeworkProblem}[Problem 6]

    \textbf{(Exercise 3.14)}
    Write prompt

    \subsection{Answer}

    TODO

\end{homeworkProblem}

\pagebreak

\nobreak\extramarks{Problem 7}{}\nobreak{}

\begin{homeworkProblem}[Problem 7]

    \textbf{(Exercise 3.17)}
    Write prompt

    \subsection{Answer}

    TODO

\end{homeworkProblem}

\pagebreak

\nobreak\extramarks{Problem 8}{}\nobreak{}

\begin{homeworkProblem}[Problem 8]

    \textbf{(Exercise 3.22)}
    Write prompt

    \subsection{Answer}

    TODO

\end{homeworkProblem}

\end{document}